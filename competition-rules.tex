\section{The Competition Rules}\label{app:competition-rules}

This appendix describes additional procedures for Small Size League matches.

\subsection{Extra Time}
If the game is drawn after the end of the second period and the game needs to end with a clear winner, extra time will be played (as stated in laws 7 and 10).
Before the first half of extra time, there will be an interval that must not exceed 5 minutes.

\subsubsection{Periods of Play in Extra Time}
The extra time lasts two equal periods of 5 minutes, unless otherwise mutually agreed between the referee and the two participating teams.
Any agreement to alter the periods of extra time (for example, to reduce each half to 3 minutes because of a limited schedule) must be made before the start of play and must comply with competition rules.

\subsubsection{Extra Time Half-Time Interval}
Teams are entitled to an interval at half-time.
The half-time interval must not exceed 2 minutes.
The duration of the half-time interval may be altered only with the consent of both teams and the referee.

\subsubsection{Timeouts}
Each team is allocated two timeouts at the beginning of extra time.
A total of 5 minutes is allowed for all timeouts.
The number of timeouts and the time not used in regular game are not added.
Timeouts in extra time follow the same rules as in regular game (stated in \autoref{sec:duration-of-the-match}).

\subsection{\removed{Penalty }Shoot-Out}
\removed{If the game is drawn after the end of the second period of extra time, kicks from the penalty mark will be taken to decide which teams wins the game.
The penalty mark for the Penalty Shoot-out procedure is located 6 meters away from the defending goal line, in the middle between the touch lines.}
\added{If the game is drawn after the end of the second period of extra time, an one on one challenge will decide which team wins the game.}


\subsubsection{Preparation}
Before the \added{shoot-out starts}\removed{first penalty is kicked}, there will be an interval that must not exceed 2 minutes.
\removed{This time is suggested to be used by the teams in dialogue with the referee and his assistants to check whether the goalkeeper's position is correct (on the line) and all other rules for penalty can be full filled as stated in \autoref{sec:penalty-kick}.}
The referee determines (e.g., by flipping a coin) which team defends which goal as well as which team \added{starts attacking first}\removed{has to take the first penalty kick}.

\subsubsection{Procedure}
\added{
\begin{itemize}
\item
A maximum of 2 robots per team are allowed on the field in order to avoid interference
\item
If two robots of the same team are on the field, one of them has to be located within 1 meter distance from the attacking team's
goal line.
\item 
The ball is placed 8 meters away for Division A and 6 meters away for 
Division B from the defending goal line, in the middle between the touch lines for each attack
\item
Both teams alternately attempt to score a goal until each team has performed 5 attempts.
\item
The attacking robot has 10 seconds to score on the opponent's goal.
After 10 seconds, if the ball leaves the field or if two robots of the attacking team touch the ball during one attempt, the attempt is
considered no goal.
\item
The attack is stopped and considered no goal, when the keeper touched the ball (a bounce on the robot case is not counted as touching)
\item
The attacker may only play the ball forward (with respect to the x-axis of the field) towards the goal 
\item
The rules of \autoref{sec:fouls-and-misconduct} apply. If the attacker team has violated a rule, the attempt is considered no goal. Otherwise, the attempt is repeated
\item
In the case of a repeated infraction of the defending team, a goal is awarded to the attacking team.
\item
No timeout is possible during the shoot-out
\item
Robots may be exchanged between the kicks following the interchange
rules of \autoref{subsubsec:number-of-robots-interchange}, with the exception
that the interchange robot may immediately enter the field after the robot which
is to be replaced has been removed from the field, independently from its
location.
\item
As switching sides would cost too much time both goals are used.
\item
If a decision is reached for one team, the attempts are stopped by the referee.
\item
If, after 10 attempts, no decision is reached, each team takes another attempt in the same order as before.
This procedure (one attempt for each team) is continued until a decision is reached.
\end{itemize}
}

\removed{
During the kicks from the penalty mark, a maximum of 2 robots per team is on the field in order to avoid interference.
The kicks from the penalty mark are taken alternately by the teams until each team has kicked 5 penalties.
If a decision is reached for one team, the kicks are stopped by the referee.
For all penalties, the rules of \autoref{sec:penalty-kick} apply.
A second kick (e.g., if the ball bounces back from the goalkeeper or a goalpost) or a bounce back from the kicker will not score; as soon as the kicker touches the ball after he released it the first time the penalty is over.
During the kicks from the penalty mark no timeout is possible.
Robots may be exchanged between the kicks following the interchange rules of \autoref{sec:number-of-robots}.
If two robots of the same team are on the field, one of them has to be located within 1 meter distance from the attacking team's
goal line.
Both teams alternately attempt to score
a goal until each team has performed 5 attempts. If a decision is reached for
one team, the attempts are stopped by the referee.
For each attempt, the defending goalie starts in a position in which it is
touching the goal line.
The ball is located 6 meters away from the defending goal line, measured
perpendicularly from the center of the goal.
The attacking robot has 10 seconds to score on the opponent's goal.
After 10 seconds, if the ball leaves the field or if two robots of the attacking team touch the ball during one attempt, the attempt is
marked as a no-goal.
In the event of a rule infraction as described by the rules of \autoref{sec:fouls-and-misconduct},
the attempt is repeated.
In the case of a repeated infraction of the defending team, a goal is
awarded to the attacking team.
In the case of a repeated infraction of the attacking team, the attempt
is marked as a no-goal.
Robots may be exchanged between the kicks following the interchange
rules of \autoref{subsubsec:number-of-robots-interchange}, with the exception
that the interchange robot may immediately enter the field after the robot which
is to be replaced has been removed from the field, independently from its
location.
}

\removed{
As switching sides would cost too much time and would force the teams to touch their systems both goals are used.
If after 10 kicks no decision is reached, each team takes another penalty in the same order as before.
This procedure (one penalty each team) is continued until a decision is reached.
}

\subsection{Abandoned Match}
If one of the teams abandons the match, before or during its course, the opponent will be awarded winner for all purposes.
However, solely for the purpose of goal difference counting, the winner team can, at its decision, continue to play by itself, and the goals scored will continue to be computed.

If the two teams abandon the match, before or during its course, both teams are considered to have lost the match.
Abandoned matches cannot result in ties.

The competition records will indicate the team(s) that abandoned the match.

A team that refuses to make a good faith effort to participate in a
scheduled game will be disqualified from the competition.

\subsection{Early Termination of Match at Score of 10}

When the score difference reaches 10 goals in a round-robin (not tournament) game, the match is automatically terminated and the team with more goals declared the winner.

\subsection{Round-Robin Ranking Criteria}

During the round-robin phase of the competition, the ranking of each team in each group will be determined by the following criteria, in order:
\begin{itemize}
\item greatest number of points obtained in all group matches
\item goal difference in all group matches
\item greatest number of goals scored in all group matches
\end{itemize}

\subsubsection{Tiebreaking}
If two or more teams are equal on the basis of the above criteria, the tiebreaking procedure to determine their rankings will be determined by the following criteria, in order:
\begin{itemize}
\item greatest number of points obtained in the group matches between the teams concerned
\item goal difference resulting from the group matches between the teams concerned
\item greater number of goals scored in all group matches between the teams concerned
\item drawing of lots by the Organising Committee
\end{itemize}

\added{%
\subsection{Divisions}\label{sec:divisions}
The Small Size League is divided into two divisions with separate tournaments: Division A and Division B. Division A is aimed at advanced teams whereas new and/or less competitive teams can play in Division B. Each team will only play in one of those two divisions. When submitting the qualification material, the team also chooses a preferred division including a short rationale. The OC will have the final word.
}
